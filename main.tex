\documentclass{resume} % Use the custom resume.cls style
\usepackage[left=0.75in,top=0.6in,right=0.75in,bottom=0.6in]{geometry} % Document margins
% \usepackage[left=1in,top=1in,right=1in,bottom=1in]{geometry} % Document margins
\name{MU-TI CHUNG\vspace{-0.3em}} % Your name
%\address{3F., No.38, Ln. 308, Guangfu S. Rd., Da’an Dist., Taipei City 106, Taiwan (R.O.C.)} % Your secondary addess (optional)
\address{(+886)~979~$\cdot$~442~$\cdot$~550~ $|$ mtchung037@gmail.com} % Your phone number and email
% \address{https://sites.google.com/view/muti-chung} % Your address

\begin{document}
%---------------------------------------------------------------------------
%	SUMMARY SECTION
%---------------------------------------------------------------------------
\begin{rSection}{Summary}
Machine Learning Engineer specialized in model training, compression, and optimization for AI accelerators. Proven ability to build production-grade Python libraries and training pipelines for large-scale model fine-tuning and compression (CNNs, LLMs). Skilled in distributed training and hardware-aware algorithm design, with a track record of deploying tools that significantly reduce model size while maintaining accuracy.
    % Demonstrated success in shipping tools adopted by multiple teams compressing models to a fraction of their original size while maintaining accuracy.
\end{rSection}

%---------------------------------------------------------------------------
%	TECHNICAL STRENGTHS SECTION
%---------------------------------------------------------------------------

\begin{rSection}{Technical Skills}
    \vspace{-1em}
    \item[] \textbf{Programming Languages}: Python, C/C++, Rust.
    \item[] \textbf{Frameworks \& Libraries}: PyTorch, TensorFlow, Transformers, TRL, Accelerate, DeepSpeed, Triton.
    % \item[] \textbf{Tools}: Git, Jupyter, Sphinx.
    % \item[] \textbf{Languages}: English (fluent, GRE: V156 Q170 AW 3.5; TOEFL: R30 L30 S23 W27), Mandarin (native)
\end{rSection}

%---------------------------------------------------------------------------
%   WORK SECTION
%---------------------------------------------------------------------------

\begin{rSection}{Work Experience}
    \textbf{Ambarella} \hfill Jul. 2021 - Current \\
    \textit{Software Engineer} \hfill \textit{Hsinchu, Taiwan} \\
    \begin{rSubsection}{Cross-Framework Pruning Library}{}{}{}
        \item Led design and development of a framework-agnostic model pruning library, enabling unified APIs and features across TensorFlow 1/2 and PyTorch.
        \item Implemented structured and unstructured pruning using L1/L2-norm criteria, with support for gradual pruning schedules and layerwise sparsity analysis with respect to hardware constraints.
        \item Achieved 50–90\% model sparsity with $<$1\% accuracy loss, allowing hardware acceleration on SoCs.
        \item Complete CI support built: unit tests, end-to-end regression tests, and Sphinx-based documentation with runnable Jupyter examples.
        % \item Promoted Sphinx adoption across teams, replacing outdated documentation workflows.
        % \item Evaluated DLPack, CuPy, and Numba to accelerate tensor operations beyond NumPy.
        % \item Presented work to VP-level leadership; library now serves as backend to user-facing pruning tools.
    \end{rSubsection}

    \begin{rSubsection}{Compression-Aware Training for LLMs}{}{}{}
        \item Developed compression-aware training library with support for DeepSpeed ZeRO-3 to enable VRAM-efficient fine-tuning of LLMs.
        \item Integrated a wide range of post-training and training-time compression techniques, including GPTQ, SparseGPT, Wanda, QAT, OmniQuant, LoRA/qLoRA, knowledge distillation.
        \item Successfully compressed LLaMA models to 12.5\% of original size with minimal accuracy degradation.
        \item Researched and created deployment guides for clients with limited compute budgets, covering training strategy, compression trade-offs, and hardware-aware optimization.
        \item Worked with tools such as Triton and liger-kernel to optimize training throughput and resolve memory bottlenecks.
    \end{rSubsection}

    \begin{rSubsection}{Activation Sparsity in LLMs}{}{}{}
        \item Independently led research on activation sparsity in LLaMA-2, Mistral-7B, and Qwen-2.5, achieving 70–90\% sparsity in MLP layers via continual pretraining.
        \item Integrated methods like ReLUfication, Deja Vu, TurboSparse, and Q-Sparse using PyTorch and Triton.
        \item Developed a predictor network to identify active MLP channels at inference, enabling selective weight loading and I/O-bound acceleration with minimal accuracy loss.
        % \item Delivered a modular, near-production quality codebase for internal use.
        % \item Independently led activation sparsity research targeting inference-time acceleration for LLMs (LLaMA-2, Mistral-7B, Qwen-2.5).
        % \item Achieved 70–90\% activation sparsity in MLP modules via continual pretraining, focusing on compute-heavy, I/O-bound submodules for maximal efficiency gains.
        % \item Implemented and benchmarked recent research methods including ReLUfication, Deja Vu, TurboSparse, and Q-Sparse, using PyTorch and custom Triton kernels where necessary.
        % \item Developed a predictor network (inspired by MoE routing) to identify active MLP channels at inference time, enabling selective weight loading and substantial I/O reduction. Balanced predictor precision/recall vs. model accuracy, enabling runtime gains without degrading performance.
        % \item Maintained a clean, modular research codebase approaching production quality, with reusable components and well-managed Git history.
    \end{rSubsection}
\end{rSection}


%---------------------------------------------------------------------------
%	EDUCATION SECTION
%---------------------------------------------------------------------------

\begin{rSection}{Education}
    \textbf{University of Michigan} \hfill Sep. 2019 - Dec. 2020$\ $\\
    \textit{M.S. in Robotics} $|$ GPA: 4.0/4.0 \hfill \textit{Ann Arbor, MI}
    % Course: Computer Vision, Deep Learning for Vision, Mobile Robotics, Self Driving Car, Robotic System Lab, Math for Robotics, Convex Optimization in Control, Foundations of Artificial Intelligence

    \textbf{National Taiwan University} \hfill Sep. 2014 - Jun. 2018$\ $\\
    \textit{B.S. in Mechanical Engineering} $|$ GPA: 4.13/4.3 \hfill \textit{Taipei, Taiwan}
    % Course: Digital Control System, Linear Control System, Optimization in Engineering, Automatic Control, System Dynamics, System Identification

% \begin{rSubsection}{University of Michigan}{Sep. 2019 - Dec. 2020}{M.S. in Robotics $|$ \normalfont{GPA: 4.0/4.0}}{Ann Arbor, MI}
%     \item Course: Computer Vision, Deep Learning for Vision, Foundations of Artificial Intelligence, Mobile Robotics, Convex Optimization in Control, Self Driving Cars, Robotic System Lab, Math for Robotics
% \end{rSubsection}

% \begin{rSubsection}{National Taiwan University}{Sep. 2014 - Jun. 2018}{B.S. in Mechanical Engineering $|$ \normalfont{GPA: 3.96/4.0}}{Taipei, Taiwan}
%     \item Course: Digital Control System, Linear Control System, Optimization in Engineering, Automatic Control, System Dynamics, System Identification
%     % \item GPA of Last 60 Credits (Last 2 Years): 4.0/4.0
%     % \item Rank: 7/145
% \end{rSubsection}
\end{rSection}

% %---------------------------------------------------------------------------
% %	HONORS & AWARDS SECTION
% %---------------------------------------------------------------------------

% \begin{rSection}{Honors $\&$ Awards}

% \begin{rSubsection}{NTU Presidential Awards}{3 times, 2015 - 2018}{}{}
% \item Given to top 5\% of class.
% \end{rSubsection}

% \begin{rSubsection}{Altruistic Award, College of Engineering, NTU}{Jun. 2017}{}{}
% \item Given each academic year to students for devoting oneself to public services.
% \end{rSubsection}

% \end{rSection}

%---------------------------------------------------------------------------
%	PUBLICATION SECTION
%---------------------------------------------------------------------------
% \begin{rSection}{Publication}
% \item[{[1]}] Tsung-Yen Tsai, Chia-Yu Chang, \textbf{Mu-Ti Chung}, Kuan-Chieh Lu, Jhih-Fong Huang, and Wen-Pin Shih, “A Portable Exoskeleton Driven by Pneumatic Artificial Muscles for Upper Limb Motion Replication,” \textit{2018 ICCMA International Conference, Tokyo, Japan, Oct. 2018}
% \item[{[2]}] Kuan-Chieh Lu, \textbf{Mu-Ti Chung}, Heng-Sheng Chang, Wei-Ting Chien and Wen-Pin Shih, “Control Strategy of Robotic Arm Based on Antagonistic Pneumatic Artificial Muscles,” \textit{2017 ICIUS International Conference, Taipei, Taiwan, Aug. 2017}

% \end{rSection}

%---------------------------------------------------------------------------
%	PROJECT EXPERIENCE SECTION
%---------------------------------------------------------------------------
% \begin{rSection}{Course Projects}
%     % EECS 504
%     \begin{rSubsection}{Virtual Footwear Fitting Room}{Sep. 2020 - Dec. 2020}{}{}
%         \item Built footwear-swapping pipeline to provide visualization of user images with retailer shoes on.
%         \item Implemented YoloV3 model with PyTorch and trained on Open Images Dataset V6 to detect footwears.
%         \item Applied min-cut/max-flow graph cut algorithm for segmenting user footwear from background.
%         \item Estimated homography transformation with PCA and RANSAC for footwear stitching. 
%     \end{rSubsection}

%     % EECS 598 DLCV
%     \begin{rSubsection}{Deep Learning for Vision}{Sep. 2020 - Dec. 2020}{}{}
%         \item Reproduction of prominent architectures including ResNet, Yolo, Faster R-CNN, LSTM and GAN.
%         \item Implementation with Pytorch and trained on datasets such as CIFAR-10, COCO, PASCAL VOC, etc.
%     \end{rSubsection}

%     % ROB 530
%     \begin{rSubsection}{InEKF Localization and Semantic Mapping on KITTI Dataset}{Jan. 2020 - Apr. 2020}{}{}
%         \item Implemented invariant extended Kalman filter (InEKF) with IMU and GPS data on KITTI dataset to improve state estimation and localization accuracy.
%         \item Performed image segmentation with U-Net architecture to semantically label LiDAR point cloud.
%     \end{rSubsection}
    
%     % ROB 550
%     \begin{rSubsection}{Robotics Arm with Block Detection}{Sep. 2019 - Dec. 2019}{}{}
%         % \item Developed 6-DOF robot arm and Kinect-based camera system to pick up color blocks.
%         \item Designed end effector for 6 DOF robotics arm and manipulated it with forward/inverse kinematics.
%         \item Performed camera calibration and block detection with OpenCV in python.
%         \item Synthesized block detection and motion planning to complete tasks of block grasping and stacking.
%     \end{rSubsection}

%     \begin{rSubsection}{Autonomous SLAM Mobile Robot}{Sep. 2019 - Dec. 2019}{}{}
%         \item Implemented SLAM on robots in C++ with Raspberry Pi, LIDAR and IMU.
%         \item Performed particle filter based localization with odometry motion model and likelihood sensor model. 
%         \item Navigated through unknown areas in maze with A* planner and exploration algorithm.
%         \item Integrated camera-and-arm system with robot to perform block hunting in maze.
%     \end{rSubsection}

%     % \begin{rSubsection}{3DOF Inverted Pendulum Self-balancing Robot}{Sep. 2019 - Dec. 2019}{}{}
%     %     \item Designed controller with IMU and encoder feedback in C to realize self-balance and pose control.
%     %     \item Integrated motor and balancing control and estimated robot position with odometry fusion algorithm.
%     %     \item Designed trajectory profiles for tasks including driving square, drag race and path navigation.
%     % \end{rSubsection}

%     % % ROB 535
%     % \begin{rSubsection}{Racing on Pre-Defined Map with Unknowned Obstacles}{Sep. 2019 - Dec. 2019}{}{}
%     %     \item Designed online trajectory planner to follow pre-defined track and avoid obstacles known at run-time.
%     %     \item Implemented quadratic programming based model predictive control (MPC) on bike model in MATLAB for trajectory following.
%     % \end{rSubsection}

% \end{rSection}
% \newpage

% \begin{rSection}{Course Projects}
%     \begin{rSubsection}{Mobile Robotics}{Jan. 2020 - Apr. 2020}{}{}
%         % \item[]
%         \begin{rSubsection}{InEKF Localization and Semantic Mapping on the KITTI Dataset}{}{}{}\vspace{-0.25em}
%             \item Implemented InEKF with IMU and GPS data on KITTI dataset to improve state estimation.
%             % \item Performed image segmentation with U-Net architecture to semantically label LiDAR point cloud.
%         \end{rSubsection}
%     \end{rSubsection}

%     \begin{rSubsection}{Robotic System Lab}{Sep. 2019 - Dec. 2019}{}{}
%         \item[]
%         \begin{rSubsection}{ArmLab}{}{}{}\vspace{-0.25em}
%             \item Developed 6-DOF robot arm and kinect-based camera system to pick up color blocks.
%         \end{rSubsection}\vspace{-0.5em}
        
%         \begin{rSubsection}{BalanceBot}{}{}{}\vspace{-0.25em}
%             \item Designed controller scheme for two-wheel robot to realize self-balance and pose control.
%         \end{rSubsection}\vspace{-0.5em}

%         \begin{rSubsection}{BotLab}{}{}{}\vspace{-0.25em}
%             \item Utilized particle filter based SLAM and A* path planning algorithm on a mobile robot for map exploration.
%             \item Combined camera-and-arm system and the robot together to perform block hunting in mazes.
%         \end{rSubsection}
%     \end{rSubsection}

%     % \begin{rSubsection}{Self Driving Car}{Sep. 2019 - Dec. 2019}{}{}
%     %     \item[] 
%     %     \begin{rSubsection}{Racing on a Pre-defined Map with Unknowned Obstacles}{}{}{}\vspace{-0.25em}
%     %         \item Designed online trajectory planner to follow a pre-defined race track and avoid obstacles known only at run-time.
%     %         \item Implemented quadratic programming based model predictive control on a bicycle model in MATLAB for trajectory following.
%     %     \end{rSubsection}
%     % \end{rSubsection}

% % \begin{rSubsection}{Motor Control Simulation}{Sep. 2017 - Jan. 2018}{Digital Control System, ME5247}{}
% % \item Analyzed the math model of the motor; implemented command feedforward (CFF), disturbance input decoupling (DID) and observer control strategies on the motor model with Simulink.
% % \end{rSubsection}

% % \begin{rSubsection}{Tracing Optimization Project}{Sep. 2017 - Jan. 2018}{Optimization in Engineering, ME7129}{}
% % \item Derived the math model of the line-tracing dynamics and optimized its performance with several optimization algorithms.
% % \end{rSubsection}

% % \begin{rSubsection}{Air-Propelled Line Tracing Car}{Feb. 2017 - Jun. 2017}{Practice of Mechanical Engineering, ME1011}{}
% % \item Built the mechatronics system of the car; devised a safety circuit to ensure stability.
% % \item Developed gain scheduling control strategies to improve tracking robustness and programmed a software on the microcontroller.
% % \end{rSubsection}

% % \begin{rSubsection}{Pole Climbing and Metal Ball Collecting Robot}{Sep. 2016 - Jan. 2017}{Machine Design Theory, ME3004}{}
% % \item Designed the mechanisms of the robot and selected suitable mechatronic components; resolved the key issue of insufficient current of signal receiver by implementing a regulator circuit.
% % \item Ranked first place in the class’ final competition.
% % \end{rSubsection}

% \end{rSection}



%---------------------------------------------------------------------------
%	RESEARCH EXPERIENCE SECTION
%---------------------------------------------------------------------------
% \begin{rSection}{Research Experience}
%     \begin{rSubsection}{Run-to-Run Control for Additive Manufacturing Processes}{Jan. 2020 - Aug. 2020}{Advisor: Prof. Kira Barton}{Ann Arbor, MI}\vspace{-0.25em}
%         \item Designed run-to-run (R2R) control scheme and iterative learning controller (ILC) for fused deposition modeling (FDM) systems to improve accuracy in bead cross-sectional dimensions. Developed simulation environment in MATLAB to model and control the layer-to-layer dynamics of FDM systems.
%     \end{rSubsection}
    
%     % \begin{rSubsection}{Control of Pneumatic Artificial Muscle Driven Exoskeleton}{Mar. 2017 - Jun. 2018}{Advisor: Prof. Wen-Pin Shih}{Taipei, Taiwan}\vspace{-0.25em}
%     %     \item Developed a database of the pneumatic artificial muscle and implemented command feedforward control (CFF) to improve tracking performance; designed microcontroller programs to replicate user motion for exoskeleton control.
%     % \end{rSubsection}
% \end{rSection}


%---------------------------------------------------------------------------
%	WORK EXPERIENCE SECTION
%---------------------------------------------------------------------------
% \begin{rSection}{Work Experience}
%     \begin{rSubsection}{Gogoro Inc.}{Mar. 2018 - Jun. 2018}{R\&D Intern in Powertrain Team}{Taoyuan, Taiwan}\vspace{-0.25em}
%         \item Developed data processing algorithm to filter oscillating sensor signals.
%         \item Researched in Traction Control System (TCS) for electric scooters; built observer with recursive least square to estimate the slipping conditions of the vehicle.
%     \end{rSubsection}
% \end{rSection}



%---------------------------------------------------------------------------
% %	Extracurricular Activity
%---------------------------------------------------------------------------

% \begin{rSection}{Extracurricular Activity}
% \begin{rSubsection}{Maintenance and Management Team of NTUME Makerspace}{Sep. 2016 - Jan. 2018}{}
% \item 
% \item Conducted maintenance and repair of facilities; instructed students in operations of equipments.
% \end{rSubsection}

% \begin{rSubsection}{Student Association of NTUME}{Sep. 2015 - Jun. 2017}{President}{}
% \item Responsible for the communication between professors, departmental staffs and undergraduate students, particularly on the issues on student rights and activities.
% \end{rSubsection}
% \end{rSection}



%---------------------------------------------------------------------------

\end{document}
